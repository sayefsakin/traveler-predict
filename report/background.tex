\section{Related Works}
\label{sec:background}
Large-scale web-based pan/zoom visualizations developed by Tao et al.\cite{tao2019kyrix} is more relevant to my project. Their technique was mostly to
predict next pan/zoom location which required less intensive learning based techniques. They developed a declarative language for easy specification of user
behavior and data management model. Battle et al.\cite{battle2019role} did thorough investigation of explorative data analysis by showing comparison between
latency, task complexity, and user performance. An earlier work\cite{battle2014dynamic} from the same author analysed different statistical characteristics
(e.g. histogram) among multiple user interactions over temporal dataset. They also proposed a general purpose tool, \emph{ForeCache}\cite{battle2016dynamic},
for exploratory browsing of large dataset. By learning first from the user's movements, they prefetched the data using different statistical data
characteristics to find similarity to the user's past behavior. Their data model and the prediction engine design has been considered a baseline for
measurement in visual data exploratory domain. Liu et al.\cite{liu2013immens} proposed a specialized data structure called multivariate data tiles for
dynamic loading and pre-processing of data. They also synthesis the data for better representation and processing parallelization. Huang el al.
\cite{huang2018optimization} \cite{huang2019aideme} used active learning framework to leverage the subspatial context and conjunctive context of database queries. Their
developed
dual space model algorithm able to bypass the slow convergence problem of active learning techniques. In my project, most of their proposed mechanisms have
been adapted due to the high similarity of the problem domain and use cases. \emph{Skyrec}\cite{liu2019improvement} of Liu et al. used query session summary
method to ease-up the query recommendations. They also developed a frontend interface named \emph{SkyServer Surfliner} for interactive recommendation.