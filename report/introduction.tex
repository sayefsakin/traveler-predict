\section{Overview}
\label{sec:overview}
Large-scale parallel applications have become the backbone of a wide variety of scientific and technological research\cite{ashby2010opportunities,
hoisie2012ascr}. Deciphering features from the runtime system of these appications as well as optimizing them is a significant and largely unsolved problem\cite{daly2011tools}. Optimizing resources related to time-oriented process scheduling is essential for productivity. Time series visualization provides a
practical medium for exploratory visual data analysis of these parallel systems\cite{aigner2011visualization}. Typically, high-end scientific
simulations require time consuming and intensive operations, incurring massive temporal dataset to decipher. Interactive visual data exploration
of these massive temporal dataset is becoming increasingly challenging due to the rapid generation of data, in the order of terabytes or petabytes.
Therefore, an automated feature recommender system of a temporal dataset is essential for exploratory data analysis. It will also give clues on how different
hardware metrics change over time, which could be valuable information for scientists required to run, analyze, and tweak high-end time consuming simulations.
\par
Phylanx\cite{tohid2018asynchronous} is a general purpose computation system of distributed arrays to enhance operations over large-scale distributed system. It
is an actively-developed system to provide the benefits of distributed resources for data scientists. Phylanx first translates the code into PhySL, an intermediate
representation in a domain-specific, functional language. The function calls, control flow operations, data operations, and blocks found in the PhySL
representation are referred to as primitives. These primitives are translated to tasks in the HPX, a C++ standard library and asynchronous tasking runtime.
The dependencies of each primitive are the arguments it needs to execute (which may be other primitives), data access operations, or any other constraints on
variables the primitive uses. Temporal execution information of Phylanx is printed on a trace file and visualized using Traveler.
\par
The Traveler \cite{Traveler2019} is an integrated graph visualization system for asynchronous multi-task execution. It uses
OTF2\cite{otf2_developer_community_2019_3356709} \cite{knupfer2012score} trace data generated from Phylanx execution, parse them, and represent them in
different interactive views\textemdash Gantt view, Utilization view, Tree view, and Time series view, etc. Due to the massive number of temporal data, the
loading time of these views becomes slower and conspicuous lagging becomes apparent during interactive inspection over these views. The goal of my project is
the identification and discovery of interesting features\textemdash like cache overflow, unnecessary memory overhead, CPU exhaustion or starvation,
etc .\textemdash from the temporal dataset of hardware performance metric. In the aim of achieving this goal, an automated learning based recommender system
has been developed to predict and recommend interesting features which will be useful for further analysis of the runtime system (the trace data it is used
to learn from). The contributions of my project are summarized as follows:
\begin{itemize}
    \item Recommender Engine modeling - The dual space model (DSM)\cite{huang2018optimization} proposed by Huang et al. has been adapted to design a
    recommender engine.
    \item Feature abstraction - Temporal data of a performance metric from different hardware locations have been converted into individual input features for
    the recommender engine.
    \item Data adapter - An intermediate parser has been designed to receive data from the Traveler platform to the recommender engine, parse them and
    sent back through a convenient interface.
    \item Performance evaluation - An elementary performance evaluation has been conducted to assess the practicality and feasibility of the designed
    recommender model.
\end{itemize}
