\section{Evaluation}
\label{sec:evaluation}
How did you evaluate.
%The evaluation is conducted in two phases. In the first phase, a short presentation of this project was given in front of some students and their feedback was
%collected over an online form. In the second phase, an email was sent to the Phylanx developers, containing some screenshots of the visualization and their
%feedback was collected over email. Both of these evaluation phases are described in the following two subsections.
%%
%%
%\subsection{Non-expert Feedback}
%A short presentation about this project was delivered with some visual demonstration (with screenshots) in front of a set of participants. The participants of
%my presentation are 10 graduate students from the department of computer science, applied mathematics, lunar \& planetary lab of University of Arizona. During
%the presentation, I covered both the backend data handling and frontend visual design and representation methodologies after providing some background
%knowledge of Phylanx and traveler-integrated project. With the help of the course instructor, an online form was presented to the participants and feedback
%was collected from each of them. The participants were asked about what they think the strength of my project and how it could be improved.
%\par
%Most of the participants found the visualization intuitive as it works with temporal data and presents that in conventional way. Some of them liked the
%juxtaposed views and interactive features over multiple views like zoom in/out, mouse hover, etc. Two participants also commented that, my visualization
%makes it easier and clear to find points where CPU utilization cycles are changing which will help in debugging. There is also one participant who couldn't
%understand my visualization as the participant mentioned the presentation wasn't clear enough.
%\par
%The suggestions given by the participants were very informative and helpful toward the future direction of my project. Most of the participants suggested to
%use color encoding based on some properties to increase the identification of individual line. One mentioned to use pop-out functionality to highlight or
%isolate data points. Some pointed out to use clustering methods for different location to make the visualization less cluttered. On participant advised to
%use binary search tree during data bundling instead of the interval tree.
%%
%%
%\subsection{Expert Evaluation}
%Since I worked with OTF2 trace data generated from Phylanx execution on parallel machines, I reached out to Phylanx team members, via email, for their feedback
%on how my visualization can augment into their debugging and artifact isolation process. I included several screenshots of my visualization with relevant
%description of the functionalities. Two of the principal investigators of Phylanx replied to that email and commented on the visualization.
%\par Both of the principal investigators praised the newly developed line chart, after some back-and-forth emailing with the
%functionality description. One of them pointed out that, the core design is useful to represent the metric PAPI\_L1\_DCM (level 1 data cache misses), while
%in my design, I used the metric PAPI\_TOT\_CYC (cpu cycle). Also a bug on the trace data\textemdash which prints the same metric values for consecutive
%enter/leave events\textemdash was manifested and acknowledged by the investigators.
%%
%%
%\subsection{Discussion}
%It was clear from the evaluation that the newly developed line chart in association with other views (Gantt view, Utilization view) augments in understanding
%of the trace data. Though only one metric is used, the core design of the visualization is useful to represent different hardware performance metric values.
%It also becomes intuitive to track down primitive and its related metric values because of the juxtaposed view design. Small tweaking on the gantt view, like
%highlighting similar intervals with same GUID and primitive name helped to instrument the dependencies among them.
%\par Several interactive features like zoom in/out, panning increase the readability of the line chart which makes the frontend more effective for
%representing traces from longer running program executions. Also the usage of interval tree on the backend with some performance tweaking on histogram
%calculation make it more adaptable to higher number of input trace data. These design choices make the visualization comparatively more scalable with respect
%to the previous implementation of the traveler project. Though the visual design and the backend implementation hasn't ready for adapting into other parallel
%system (than Phylanx), it gives a proper indication on how to do so\textemdash with convenient API design and backend data parsing would easily make it
%adaptable to other distributed programs.
%\subsection{Limitations}
%The target population (domain expert) for this project is really small\textemdash only limited to the team members of Phylanx. Since the traveler project is an
%actively developed project, it has some complex procedures on live deployment and dependency on over several other modules to
%become ready. Because of this, a live demo for the limited number of Phylanx experts was not possible during the evaluation process. Though it could be done
%with remote desktop sharing methods, it will not be convenient for the participants. Again, while collecting feedback with email, the description of the
%views was not clear and concise. A formatted description with related questionnaries and prerecorded video demo would make the evaluation over the email more
%convenient.
%\subsection{Future Evaluations}
%In future, more formatted evaluation will be conducted through exhaustive feedback procedures to make the visualization more effective for the Phylanx
%members. A live demo will be prepared by deploying the project and make it accessible for the participants. A set of predefined questions will be prepared
%for the evaluation process. The questions will be presented and feedback will be collected via online forms, so that the participants can submit the forms in
%their convenient time just by testing the live demo. Relevant logging system will be maintained to track down the interaction of the participants over the
%visualizations.