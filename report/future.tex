\section{Future Direction}
\label{sec:future}
There are several limitations and pending modifications exist in my project which will be a good starting point for further work in the future. First of all,
an extensive user data collection should be conducted to make the prediction engine more valid and practically representative. A seperate
script can be used to either get new user input or incorporate old user data for the recommender system. Secondly, the implemented dual space model only
initiates with the labeled user data once prior to the iterations. The original DSM implementation used a probabilty function to choose between labeled
user data and DSM predicted data in successive iterations as the input for the consecutive iterations. This uncertainty sampling technique is necessary to
reduce the version space, the space of all configuration of the classification model consistent with the labeled data, for enabling an approximation of the
optimal algorithm. Thirdly, The increased number of hardware locality in a large-scale systems can easily exhaust the feature dimension. Therefore,
factorization techniques should be adapted to reduce the high-dimensional data space to a lower-dimensional spaces. Lastly, a comprehensive user study should
be conducted to evaluate the performance of the recommender system. Also, how well the prediction aligned with real life hardware artifacts can also be tracked
and used to create a validation dataset for the system.
